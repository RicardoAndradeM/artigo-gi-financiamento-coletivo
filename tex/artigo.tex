%--------------------------------------------------------------------------------------------------
% OBSERVACAO:
% 
% -> Arquivos que você pode editar:
%    - artigo.tex
%    - artigo_bibliografia.bib
%
% -> Arquivo .TeX codificado em UTF8                                                             
% -> Bibliografia em arquivo .bib (arquivo_bibliografia.bib)                                      
% -> Arquivo de imagens em .jpg, .eps ou .pdf
% -> Para compilar o TeX, execute 'compila_TEX.bat' (terminal do windows)
% 
% versão 1.1 - 19/05/2016
% versão 1.0 - 18/08/2015
%--------------------------------------------------------------------------------------------------
\documentclass{classe_cn}                 % Modelo <nao edite o arquivo classe_cn.cls>
\usepackage[brazil]{babel}                % Acentos
\usepackage[utf8]{inputenc}               % Codificação UTF8 (atenção aqui!)
\usepackage{graphicx}                     % Figura
\usepackage{amssymb}                      % Simbolos matematicos
\usepackage{color}                        % Cores
\usepackage{amsfonts}                     % Fontes
\usepackage{amsmath}                      % Fontes
\usepackage[fixlanguage]{babelbib}        % Acentos
\usepackage[normalem]{ulem}               % OK
\usepackage[retainorgcmds]{IEEEtrantools} % Formulas padrão IEEE
\usepackage{omlmathbf}                    % Simbolos Matematicos
\usepackage{epstopdf}                     % Figuras .eps
\usepackage{setspace}                     % Espaçamento flexível
\usepackage{cmap}                         % Mapear caracteres especiais no PDF
\usepackage{textcomp}                     % Funções e outros símbolos matemáticos
\usepackage{verbatim}                     % Pacotes verbatim
\usepackage{wrapfig}
%\usepackage{picins}
\startlocaldefs
\endlocaldefs

%--------------------------------------------------------------------------------------------------
% Inicio do Documento
%--------------------------------------------------------------------------------------------------
\begin{document}
\begin{frontmatter}        % Não alterar
\begin{fmbox}              % Não alterar
\dochead{Cálculo Numérico} % Não alterar

%--------------------------------------------------------------------------------------------------
% Titulo do seu Trabalho
%   - pequeno bug (nao funciona cedilha)
%   - editar manualmente o cedilha na classe_cn.cls, linha 1015.
%--------------------------------------------------------------------------------------------------
\title{O Modelo de Crowdfunding}

%------------------------------------------------
% Informações sobre o autor #1
% - Ivete Maria Dias de Sangalo
%------------------------------------------------
\author[
  addressref = {ricardo1},                 % Identifica o autor #1
  email      = {ricardo.maia@ccc.ufcg.edu.br} % email para contato
]
{
  \inits{RdAM}      % Letras iniciais do autor #1
  \fnm{Ricardo de A.}  % Nome do autor #1 (first and middle name)
  \snm{Maia}   % Ultimo nome do autor #1 (last name)
}
%------------------------------------------------
% Informações sobre o autor #2
% - James Alan Hetfield
%------------------------------------------------
\author[
  addressref = {danielle2},                      % Identifica o autor
  email      = {danielle.vieira@ccc.ufcg.edu.br} % email para contato
]
{
  \inits{DdLV}       % Letras iniciais do autor #2
  \fnm{Danielle de L.}  % Nome do autor #2 (first and middle name)
  \snm{Vieira}    % Ultimo nome do autor #2 (last name)
}
%------------------------------------------------
% Informações sobre o autor #3
% - Freddie Bulsara Mercury
%------------------------------------------------
\author[
  addressref = {damiao3},                       % Identifica o autor
  email      = {damiao.domiciano@ccc.ufcg.edu.br} % email para contato
]
{
  \inits{DRD}      % Letras iniciais do autor #3
  \fnm{Damião R.} % Nome do autor #3 (first and middle name)
  \snm{Domiciano}    % Ultimo nome do autor #3 (last name)
}
%------------------------------------------------
% Informações sobre o autor #4
% - Virinha S. Dantas
%------------------------------------------------
\author[
  addressref = {lucas4},                 % Identifica o autor
  email      = {lucas.duarte@ccc.ufcg.edu.br} % email para contato
]
{
  \inits{LVD}     % Letras iniciais do autor #4
  \fnm{Lucas V.} % Nome do autor #4 (first and middle name)
  \snm{Duarte}     % Ultimo nome do autor #4 (last name)
}
%------------------------------------------------
% Informações sobre o autor #5
% - Virinha S. Dantas
%------------------------------------------------
\author[
  addressref = {adisio5},                 % Identifica o autor
  email      = {adisio.junior@ccc.ufcg.edu.br} % email para contato
]
{
  \inits{APFJ}     % Letras iniciais do autor #4
  \fnm{Adísio P. F.} % Nome do autor #4 (first and middle name)
  \snm{Júnior}     % Ultimo nome do autor #4 (last name)
}

%------------------------------------------------
% Endereço dos autores
%------------------------------------------------
\address[id=ricardo1]{
  \orgname{Universidade Federal de Campina Grande,
           Centro de Tecnologia e Recursos Naturais,
           Unidade Acadêmica de Engenharia Civil},
  \street{Rua Aprígio Veloso, 882, Bairro Universitário},
  \postcode{58429-140},
  \city{Campina Grande},
  \cny{Brasil.}
}

\end{fmbox}

%--------------------------------------------------------------------------------------------------
% Resumo do Trabalho
%--------------------------------------------------------------------------------------------------
\begin{abstractbox}
	
\begin{abstract} 
Este artigo busca situar os conceitos de financiamento coletivo, mais conhecido pelo termo crowdfunding, O termo é usado para designar o financiamento coletivo através de sites para a concretização de projetos variados. Dessa forma, o estudo aborda diversos aspectos relacionados aos sistemas de crowdfunding, como as motivações envolvidas e sua aplicação como possibilidade de financiamento para a concretização de projetos em diferentes áreas, além das atuais plataformas de crowdfunding no Brasil e no mundo.
\end{abstract}

%--------------------------------------------------------------------------------------------------
% Palavras-chaves: Entre 3 e 6 palavras chaves
%--------------------------------------------------------------------------------------------------
\begin{keyword}
  \kwd{Financiamento coletivo}
  \kwd{Internet}
  \kwd{Inovação}
\end{keyword}

\end{abstractbox} % Não alterar
\end{frontmatter} % Não alterar

%--------------------------------------------------------------------------------------------------
% Escreva o seu artigo!
%--------------------------------------------------------------------------------------------------

%------------------------------------------------
% Seção 1
%------------------------------------------------
\section{Introdução}
O investimento em ideias é um tabu, que necessita passar por vários estágios burocráticos. As ideias devem ser apresentadas para as empresas, caso seja do interesse delas (gerando lucros) elas investirão nessa ideia. Por essa razão muitas ideias que são do interesse comum, projetos culturais sem fins lucrativos, não são interessantes para grande parte das industrias.

O Crowdfunding conseguiu modificar esse senário, pois, ele permitiu que essas ideias chegassem ao público. Basicamente o Crowdfunding é uma plataforma (não necessariamente com este nome) que permite o cadastro de campanhas para arrecadar recursos para o desenvolvimento de projetos.

\begin{quote}
“O crowdfunding cultural funciona da seguinte maneira: um grupo de pessoas é estimulado por um proponente, que inscreve seu projeto em uma plataforma de online, a investir pequenas ou médias parcelas de dinheiro a fim de alcançar um determinado orçamento, mais amplo, que objetiva viabilizar a execução de uma ação de cunho artísticocultural.”\cite[p. 3]{SEQUEIRA:sd}
 \end{quote}

Desta forma permitindo que um novo mercado surja, dando novas oportunidades, tanto para os idealizadores quanto para os beneficiados. Essa facilidade ocorre por meio dos benefícios que a internet trouxe para o homem contemporâneo, aproximando virtualmente a população mundial para contribuir para novos projetos que os instiguem a empatia. Segundo Felinto \cite{FELINTO:2012} sendo um processo que o próprio público financia um projeto através de sites da internet, promovendo para um filme, obra de arte ou produto de qualquer espécie.

Depois dessa compreensão, será demonstrado como podemos iniciar uma campanha através das ferramentas online de financiamento coletivo, que consistirá em quatro passos: a ideias, que é o início de todo o processo, onde o idealizador ou grupo possui uma ideia inicial que possui propriedades para desenvolver; O planejamento, que permitirá o crescimento e objetivos que serão alcançados; A arrecadação, qual ferramenta será usada para conseguir a quantidade necessária para pôr no mercado, ou apenas finalizar, o projeto idealizado e por último o processo de divulgação. Que chegará ao público alvo, que investirá no projeto.

Os benefícios são inúmeros como a comprovação de mercado, que consegue encontrar um padrão na preferência do público. Produção sobre demanda, que diminui a produção exagerada de produtos, entre outros que serão abordados ao decorrer deste artigo. Mas o Crowdfunding não possui apenas benefícios, que são o fracasso da campanha ou o mal-uso do dinheiro investido, que gerará consequências para o site e o idealizador do projeto.

A grandes ferramentas mundiais de financiamento coletivo como o Kickstarter que, “No ano de 2012, através do Kickstarter a cantora Amanda Palmer conseguiu arrecadar 1.2 milhões de dólares para gravar seu CD, conseguindo dez vezes mais do que o valor que o projeto pedia. ” \cite[p. 9]{CAVALCANTI:2013}, sendo uma ferramenta muito poderosa para angariar recursos. O Brasil possui plataformas de Crowdfunding, algumas que possuem um nicho específico De projetos, como o Catarse que foi o “primeiro endereço eletrônico que apresentou a plataforma de crowdfunding, voltada somente para projetos culturais” \cite{COCATE:2012}.

Em escala mundial há vários projetos bem sucedidos que forma financiado através de plataformas do Crowdfunding, como filmes, games, podcasts, livros, HQs e eventos, que serão abordados ao decorrer desse artigo.

%------------------------------------------------
% Seção 2
%------------------------------------------------
\section{Criação de campanha crowdfunfing}

Em primeiro lugar, é preciso escolher um bom motivo pelo qual esta fazendo a campanha para que convença as pessoas a apoiar o seu projeto. Pois um motivo muito fraco pode não convencer os investidores a apoiarem o projeto. Apresentando a ideia de forma clara, explicando a trajetória do projeto, dizendo o porquê dele ser importante para você e ou a sociedade, mostrando o que você pretende alcançar com a campanha.

O titulo é o primeiro encontro dos futuros contribuintes da campanha. Portanto ela deve ser explicada em poucas palavras qual o objeto da campanha e ao mesmo tempo sendo chamativa ao leitor, para que ele se convença a investi no projeto.

Oferecer uma recompensa para que as pessoas invistam no seu projeto pode ser uma boa ideia, oferecendo vantagens para certos valore investidos, para quanto mais for investido maior será os benefícios da pessoa que esta disposta a investi. Fornecer uma recompensa não é obrigatório, porém pode fazer muita diferença nos valores arrecadados para sua campanha, pois muitas pessoas investem por acreditarem no projeto e outras pelas recompensas oferecidas.

Antes de começar uma campanha de financiamento coletivo é bom saber o valor mínimo do orçamento para que seu projeto aconteça, deixando uma pequena margem para imprevistos, tentando escolher um orçamento mais realista possível.

Divulgar o máximo possível sua ideia é essencial, uma vez que quanto maior o alcance dela maior será a chance de ter pessoas dispostas a investi nela, e mesmo que não invistam elas podem se sensibilizar e acabar compartilhando sua ideia para que outras vejam, aumentando cada vez mais o alcance do seu projeto. \cite{XAVIER:2016}
%------------------------------------------------
% Seção 3
%------------------------------------------------
\section{Forças do modelo crowdfunding}

Por ser uma plataforma muito conhecida no mercado, pessoas que você nunca viu iram investi em sua campanha, podendo essa pessoa ser da sua cidade ou do outro lado do mundo, pois elas se identificaram com o objetivo da sua campanha ou se interessaram por alguma das recompensas oferecidas.

Um dos principais benefícios para quem coloca um projeto em uma campanha de crowdfunding é a comprovação de mercado. Uma vez que esse tipo de financiamento se tornou bastante atraente ultimamente para investidores, e também um bom projeto pode aumentar a reputação do autor do projeto.

Outro benefício do crowdfunding vem do fato de permitir saber exatamente quantos produtos deverão ser fabricados em uma primeira tiragem ou versão, já que só quem pagou receberá o item. \cite{PEREIRA:2016}

%------------------------------------------------
% Seção 4
%------------------------------------------------
\section{Fraquezas do modelo de crowdfunding}

O modelo está sujeito a fatores que podem desmotivar usuários. Como o não alcance da meta do orçamento do projeto na campanha, o mau uso do dinheiro obtido na campanha por partes dos donos dos projetos, entregas de produtos de baixa qualidade e atraso na entrega do produto em relação ao prazo estabelecido na campanha. \cite{PEREIRA:2016}

Outro fator que pode contribuir para o fracasso do financiamento é uma meta ou uma ideia muito ousada. Como por exemplo, o Smartphone Ubuntu Edge, que não alcançou a meta ousada de arrecadar mais de US\$ 32 milhões \cite{GAZETA:2017}

%------------------------------------------------
% Seção 5
%------------------------------------------------
\section{As plataformas de financiamento coletivo}

O crowdfunding, como já foi se tratado acima, também chamado financiamento coletivo, consiste em conseguir recursos financeiros para financiamento de um projeto a partir de doações de outras pessoas, logo abaixo será citada empresas brasileiras e estrangeiras, que conseguiram através de muitos projetos arrecadar valores impressionantes.

\subsection{Plataformas nacionais}

\subsubsection{Catarse}

Plataforma Catarse, segundo seu site que está disponível para que pessoas possa colocar seus projetos, para que possa encontrar apoiadores para o mesmo. Ela foi criada em 2011, desde seu começo até os dias atuais ela tem em seus históricos cerca de 2 mil projetos financiados por 241 mil apoiadores, tendo um lucro aproximadamente de 35 milhões arrecadado e atendendo tipos variado de projeto, diante desses valores e informações faz com que a plataforma Catarse seja a mais popular no Brasil.

A taxa administrativa da Catarse é de “13\% para campanhas de crowdfunding Flexíveis e também para campanhas Tudo ou Nada que alcançarem ou superarem a meta.” \cite{(BRASIL:2017}. Até o final de 2015, a empresa adotava a postura de “tudo ou nada”, ou seja, se o projeto não atingisse a meta de orçamento, os usuários recebiam de volta todo o dinheiro que investiram. Entretanto, agora há o Catarse Flex, que permite que o arrecadador fique com aquilo que conseguir.

\subsubsection{Kickante}

A Kickante criada em 2013, atualmente é umas das principais empresas quando se fala em arrecadação em dinheiro, tendo seu “Site recorde de arrecadação de crowdfunding na América Latina. Mais de 50.000 campanhas lançadas, captando mais de R\$ 40 milhões em colaborações.” \cite{BRASIL:2017}.

Tendo sua plataforma mais voltada para questão socail. “Aqui na Kickante, nós somos apaixonados pela cultura, causas sociais e empreendedorismo. Amamos tudo que faz uma grande diferença em nosso país, projetos grandes e pequenos que influenciem de alguma maneira a nossa comunidade.” \cite{KICKANTE:2017}. O Kickante tem como diferencial aceitar contribuições parceladas pelo cartão de crédito.

Em relação a sua taxa administrativa, “Menor Taxa do Mercado (e oferecemos assessoria gratuita). 12\% para projetos que alcançarem ou superarem a meta (Flexível ou Tudo ou Nada). Porém, se sua campanha de crowdfunding for Flexível e você não alcançar a meta, a taxa é de 17,5\%. Já inclusas as taxas cobradas pelos meios de pagamento.”  \cite{BRASIL:2017}.

%------------------------------------------------
% Seção 6
%------------------------------------------------
\section{6. As maiores plataformas de financiamento coletivo no Brasil}

\subsection{Catarse}

Considerado a primeira e maior plataforma de financiamento coletivo do Brasil, o Catarse está no ar desde o começo de 2011, já tendo agraciado mais de 2500 projetos, responsáveis por levantar mais de 44 milhões de reais investidos por mais de 280 mil pessoas.

\subsection{Kickante}

Com uma pegada de crowdfunding um pouco mais social, o Kickante já lançou mais de 19 mil campanhas responsáveis por angariar para lá de 22 milhões de reais em toda a plataforma.

\subsection{Vakinha}

Apesar de ter sido lançado oficialmente dois anos depois do Catarse, a ideia do Vakinha não é nada nova.

Criado em 2009 esse site surgiu depois que dois amigos perceberam o perrengue que era juntar a grana de um grande grupo de pessoas para bancar um projeto que no caso eram os presentes de casamento de um dos fundadores da marca.

Agora, mais de 7 anos depois, o Vakinha já conseguiu arrecadar grana para mais de 400 mil projetos cadastrados por lá.

%------------------------------------------------
% Seção 7
%------------------------------------------------
\section{As maiores plataformas de financiamento coletivo no Mundo}

\subsection{Kickstarter}

O Kickstarter é a maior plataforma do mundo: são mais de 2,4 bilhões de dólares arrecadados ao longo de 5 anos para mais de 106 mil projetos. Entre eles os famosos Óculos Rift (comprado posteriormente pelo Facebook) e até o filme baseado na série The Veronica Mars (que arrecadou mais de 5 milhões de verdinhas em sua campanha).

\subsection{Indiegogo}

Favorito dos produtos de curtas e filmes independentes, o Indiegogo é o responsável por ter dado vida a projetos como o aclamado Hardcore Henry (filme totalmente filmado em primeira pessoa com ajuda de algumas GoPros) e Lazer Team, responsável por arrecadar quase 2 milhões e meio de dólares em menos de 2 semanas.

\subsection{Rockethub}

Fundado em 2009 em Nova York, o Rockethub é um tipo de plataforma de crowdfunding que inicialmente atendia mais a projetos de tecnologia e ciência, no entanto, com o passar do tempo ele foi se transformando em um sistema mais aberto voltado a projetos educacionais que já atendeu a dezenas de milhares de instituições ao redor do mundo.

Além de ajudar a milhares de empreendedores ao redor do mundo, o financiamento coletivo também tem feito à diversão de muitos apoiadores por aí, seja através de games, discos e até filmes extremamente interessantes.

%------------------------------------------------
% Seção 8
%------------------------------------------------
\section{Ideias que já ganharam vida com esse modelo de negócio}

Com o poder de ajudar a dar vida a produtos extremamente interessantes, as plataformas de financiamento coletivo têm trazido à tona diversos projetos que talvez jamais tivessem alguma chance no mercado convencional, mas que depois de prontos muitas das vezes comprovam ser verdadeiros sucessos.

\subsection{Livros e HQs}

Lançado em maio de 2016, o projeto para o livro Contos de fadas originais, da produtora Marina Avila, pedia R\$2.950 para ganhar vida (através do Catarse). No entanto bastou pouco mais de um mês para essa quantia ultrapassar em mais de 100\% a meta, sendo um ótimo exemplo de case de sucesso literário do financiamento coletivo no país.

\subsection{Discos}

Dos capixabas do Dead Fish até os gaúchos do Apanhador Só, o que não falta no Brasil são bandas que conseguiram produzir seus discos graças ao crowdfunding.

\subsection{Filmes}

Considerado um dos maiores autores da atualidade, o norte-americano Chuck Palahniuk tem vários best-sellers em sua carreira, no entanto talvez nenhum deles tenha o mesmo sucesso de Clube da Luta, livro que teve a fama ampliada graças à versão Cult dos cinemas lá do final da década de 90.

Com um nome desse no currículo, Palahniuk não pensou duas vezes ao planejar a adaptação de outra de suas obras (Cantiga de Ninar): apostou todas as fichas em uma campanha do Kickstarter.

Em menos de 2 semanas mais de 3 mil pessoas arrecadaram os 250 mil dólares necessários para dar início a essa aguardada produção.

\subsection{Podcasts}

Considerado o podcast de maior sucesso da história, o Serial precisava de uma certa graninha para fazer a sua segunda temporada. E aí não teve jeito: lá foram os produtores recorrer à uma campanha de crowdfunding para dar vida à ela. Uma campanha que não apenas funcionou como abriu caminho para que esse modelo de financiamento fosse cogitado para uma possível próxima temporada.

\subsection{Games}

Produzido por uma parte da equipe responsável por clássicos dos games como BioShock e Dead Space (ambos com uma pegada de terror), Perception chamou a atenção da mídia internacional por causa de seu visual e sua história, que narra a vida de uma garota cega que é obrigada a andar por cenários mal assombrados.

O game precisava de 150 mil dólares para ser desenvolvido e arrecadou 168 mil e uns quebrados em pouco mais de 1 mês, se tornando um dos maiores cases de sucesso do Kickstarter.

\subsection{Shows e eventos}

Capazes de levar centenas de milhares de pessoas às ruas todo ano durante o carnaval de Belo Horizonte, o bloco Chama o Síndico em 2016 inovou ao montar uma campanha no Catarse em busca de arrecadar 25 mil reais que viabilizassem a folia nesse ano. Uma quantia que foi batida rapidamente em questão de dias.

%------------------------------------------------
% Seção 9
%------------------------------------------------
\section{Mercado mundial}

O crowdfunding já é muito bem explorado em outros países. Em 2013, o mercado mundial representou U\$ 6 bilhões, sendo que em no ano anterior, foram U\$ 2.66 bilhões. Nos Estados Unidos, um dos principais mercados de financiamento coletivo, 25\% das campanhas lançadas atingem suas metas.

O potencial de crescimento no Brasil é enorme, mas mesmo ainda sendo uma novidade, poucos empreendedores que embarcaram na onda do financiamento coletivo no Brasil conseguem fazer plataformas com diferenciais visionários e que transmitam a confiança necessária para quem investe e para quem está captando recursos para a realização de seus sonhos. Temos também a necessidade de profissionalizar o setor em nosso país. Com essa profissionalização, a tendência é aumentar o número de atletas, artistas e startups que buscam o crowdfunding para conseguir verba de maneira rápida e segura.

Nos Estados Unidos, por exemplo, o crowdfunding já é visto com um importante centro de inovação, lançando diariamente novos produtos, artes e serviços, em campanhas que chegam a coletar US\$ 14 milhões em 60 dias com o financiamento (ou pré-venda). Isso chama a atenção do mercado, de investidores anjos e até mesmo de novos consumidores para as empresas bem sucedidas em crowdfunding.

Algumas plataformas se destacam no quesito inovação e, hoje, já existem aquelas que permitem que o criador receba os fundos captados independentemente de a meta ser alcançada, através da campanha Flexível. O atual cenário econômico e o aumento de desemprego colaboraram indiretamente para a popularidade das plataformas. Os novos pequenos empresários veem nos sites de financiamento coletivo uma forma de provar a demanda por seus produtos ou serviços, partindo de investimento e riscos muito baixos. São plataformas democráticas, que permitem que qualquer pessoa, de qualquer classe, com uma boa ideia chame atenção de um público maior. Conseguindo promover ideias novas, que de outra forma não conseguiriam investimentos de empresários, que procurariam investimentos seguros, assim abrindo novas possibilidades para pessoas que não poderiam nem apresentar suas ideias para grandes empresas. Esse acesso facilitado com o público final foi uma grande virada para o mundo dos negócios.

%------------------------------------------------
% Seção 10
%------------------------------------------------
\section{Considerações Finais}

Como podemos compreender, através do texto acima, o crowdfunging possibilita o desenvolvimento de várias ideias que dificilmente conseguiriam alcançar o público com tanta clareza e rapidez. Essa “nova” abordagem do mercado de investimento proporcionou um grande aumento em projetos que antes não receberiam investimentos, através dos métodos tradicionais que é a apresentação do projeto para as empresas investirem nele. Como as empresas visam mais o seu lucro ou alguma forma de gerar marketing de seu produto para lucrarem, os projetos que elas investiriam seriam aqueles que suprissem suas necessidades. Assim, deixando projetos com características culturais ou ambientais, que não traria retorno lucrativo, eram deixados de lado, empobrecendo e desestimulando esse tipo de iniciativa.

Com a ideia de financiamento foi possível quebrar esse paradigma que impedia o investimento em projetos culturais. Com o público final podendo contribuir para ter acesso a projetos que os beneficiassem culturalmente. E não só os projetos culturais. Através dessa nova forma de arrecadação de recursos abriu um novo mercado de empreendedores que possuem ideias de proporções pequenas, mas também de ideias que não pareceriam interessantes para as indústrias, mas que deram muito certo depois de alcançarem suas metas e desenvolverem seu produto.

Com uma maior facilidade de iniciar uma campanha na plataforma de financiamento coletivo, e sua forma de divulgação podendo ser através dá própria plataforma. Mas, as redes sociais também facilitam muito essa divulgação, atraindo cada vez mais pessoas e startup com novas ideias para esse tipo de arrecadação de recursos.

%--------------------------------------------------------------------------------------------------
%--------------------------------------------------------------------------------------------------
% Define o arquivo BIB (bibliografia)
%--------------------------------------------------------------------------------------------------
%--------------------------------------------------------------------------------------------------
\bibliographystyle{bmc-mathphys}   % NAO EDITAR!
\bibliography{artigo_bibliografia} % NAO EDITAR! - Bibliography file (usually '*.bib' )

\vspace{1.0cm}

\begin{table}[h!]
\centering
\begin{tabular}{lp{8cm}}
 \raisebox{-\totalheight}{\includegraphics[width=0.3\textwidth, height=50mm]{tesla.jpg}} &  
  \vspace{0.1cm} Fulano de Tal nasceu em Campina Grande, interior do estado brasileiro Paraíba. Atualmente é
  graduando em Ciência da Computação pela Universidade Federal de Campina Grande (UFCG) e está
  cursando o terceiro semestre. Tem grande interesse em pesquisa na área Inteligência Artificial (IA),
  mais específicamente, a área de IA Distribuída (DAI). \\
  \raisebox{-\totalheight}{\includegraphics[width=0.3\textwidth, height=50mm]{tesla.jpg}} &  
  \vspace{0.1cm} Fulano de Tal nasceu em Campina Grande, interior do estado brasileiro Paraíba. Atualmente é
  graduando em Ciência da Computação pela Universidade Federal de Campina Grande (UFCG) e está
  cursando o terceiro semestre. Tem grande interesse em pesquisa na área Inteligência Artificial (IA),
  mais específicamente, a área de IA Distribuída (DAI). \\
  \raisebox{-\totalheight}{\includegraphics[width=0.3\textwidth, height=50mm]{tesla.jpg}} &  
  \vspace{0.1cm} Fulano de Tal nasceu em Campina Grande, interior do estado brasileiro Paraíba. Atualmente é
  graduando em Ciência da Computação pela Universidade Federal de Campina Grande (UFCG) e está
  cursando o terceiro semestre. Tem grande interesse em pesquisa na área Inteligência Artificial (IA),
  mais específicamente, a área de IA Distribuída (DAI).  
\end{tabular}
\end{table}





%\end{tabular}
%\end{table}

%--------------------------------------------------------------------------------------------------
% FIM DO ARTIGO
%--------------------------------------------------------------------------------------------------
\end{document}
